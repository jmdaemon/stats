\documentclass[11pt]{article}
\usepackage[noparindent]{coco}
\usepackage{thmstyles}

% TODO: Remove disgusting paragraph indent
\setlength{\lineskip}{0.0pt}
\setlength{\lineskiplimit}{0pt}
\setlength{\parskip}{0pt}
\setlength{\parindent}{0pt}
\setlength{\baselineskip}{0pt}

\geometry{
    margin=2cm
}

%\newcommand\perm[2][^n]{\prescript{#1\mkern-2.5mu}{}P_{#2}}
%\newcommand\comb[2][^n]{\prescript{#1\mkern-0.5mu}{}C_{#2}}

\newcommand{\comb}[2]{{}_{#1}\mathrm{C}_{#2}}
\newcommand{\perm}[2]{{}_{#1}\mathrm{P}_{#2}}

%\newcommand\perm[2][^n]{\prescript{{}_{#1\mkern-2.5mu}}{}P_{#2}}
%\newcommand\comb[2][^n]{\prescript{{}_{#1\mkern-0.5mu}}{}C_{#2}}

\begin{document}

% cover page
\begin{titlepage}
    \begin{center}
        \Huge Statistics
    \end{center}
\end{titlepage}

\newpage
\tableofcontents

\section{Statistics}

Statistics is the science of conducting studies to collect, organize, summarize, analyze and draw conclusions from data.

Statistics is divided into two subfields based upon how data is used:
\begin{enumerate}
    \item Descriptive statistics: Consists of the collection, organization, summarization,
        and presentation of data. No conclusions are drawn, only data described.
    \item Inferential statistics: Consists of generalizing samples from populations,
        performing estimations, and hypothesis testing, determining relationships between
        variables, and making predictions.
\end{enumerate}

\section{Data}

\subsection{Types of Data}

% TODO: Definitions for Categorical, Numerical

% TODO: Classifications definitions for:
% Categorical: Nominal, Ordinal
% Numerical: Interval, Ratio

\subsection{Categorical}

%\begin{definition}[Categorical]\label{def:categorical}
    %Qualitative / Categorical variables have some distinct categories according to some 
%\end{definition}

\subsection{Numerical}

%\section{Classifications of Data}

%For categorical there are two classifications of data...

%For numerical there are two classifications of data...

%\section{Data Collection \& Sampling Techniques}
% TODO: Classification tables

% TODO: Parameters & Statistics
% TODO: Discrete vs Continuous variables

\section{Samples}

% Define: Population, Unit, Sample
% Quick Definitions as well as helpful table for sampling techniques
% Small subsections for each sampling technique
% Define Bias

\section{Diagrams}

% Define: Peak, Unimodal, Bimodal, Multimodal, Symmetric, Uniform, Bell-shaped, Tail, Left-Right Skews
% TODO Diagrams:
% - Ogives (Cumulative Frequency Polygons)
% - Frequency Polygyons
% - Bar Graphs, Histograms, Dot plots, Box & Whisker Plots, Grouped Frequency Distributions
% - Circle Graphs / Pie Charts (Central Angles, Percentage estimations)
% - Least Squares Regression Lines (Estimation, ensuring half data points on either side of line)
% - Stem & Leaf plots

\section{Calculations}

% TODO: Sample stdev, population stddev
% TODO: Range, Class Width, Determining class limits, class boundaries, percentiles
% TODO: Mean, Median, Mode, Midrange

% TODO: Estimated mean, estimated stddev

% TODO: Regressions, Correlation Coefficient r
% TODO: PPMC, Critical Values, Degrees of Freedom
% TODO: Probability

\subsection{Shorthand Calculations}

% TODO: Regressions, stddev, etc

\section{Probability}

\begin{definition}[Event]\label{def:event}
    Collection of outcomes of a probability experiment
\end{definition}

\begin{definition}[Sample Space]\label{def:sample-space}
    The set of all possible outcome of a probability experiment.
\end{definition}

\begin{definition}[Classical Probability]\label{def:Classical Probability}
    The ratio for the number of desired outcomes of an event
    and the number of all possible events in the sample space.
\end{definition}

\textbf{NOTE} Exhaustive means that for all outcomes of an event, that set is equal to the sample space (Probability is 100\%).

\begin{definition}[Mutually Exclusive]\label{def:mutually-exclusive}
    Any two events that cannot happen simultaneously
\end{definition}

\begin{definition}[Independent Events]\label{def:independent-events}
    Two events are independent if the outcome of the first event,
    does not effect the outcome of the second.
\end{definition}

\subsection{Rules}

The domain of probabilities is $0 \leq P(A) \leq 1$ for any given event A.

The probability of the sample space is $P(S) = 1$

\begin{definition}[Addition Rule]\label{def:addition-rule}
    The addition rule for any mutually exclusive events:
    
    \begin{equation*}
        P (A \cup B) = P (A) + P(B)
    \end{equation*}

    The addition rule for any two sets:
    \begin{equation*}
        P (A \cup B) = P (A) + P(B) - P(A \cap B)
    \end{equation*}
\end{definition}

\begin{definition}[Complement Rule]\label{def:complement-rule}
    The complement of any event $A$ that is $\bar{A}$ (all events but A) is
    \begin{equation*}
        P(\bar{A}) = 1 - P(A)
    \end{equation*}
\end{definition}

\begin{definition}[Multiplication Rule]\label{def:multiplication-rule}
    The probability for any two events that occur consecutively:
    \begin{equation*}
        P (A \cap B) = P(A) \times P(B \mid A)
    \end{equation*}

    For any two independent events $A$, $B$:
    \begin{equation*}
        P (A \cap B) = P(A) \times P(B)
    \end{equation*}
\end{definition}

\begin{definition}[Conditional Probability]\label{def:conditional-probability}
    The probability that event $B$ occurs after event $A$ already occurred:
    \begin{equation*}
        P (B \mid A) = \frac{P(A \cap B)}{P(A)}
    \end{equation*}
    where $P(A) \neq 0$.

    In the case of independent events then:
    \begin{equation*}
        P (B \mid A) = P(B)
    \end{equation*}
    in other words, no information is gained from the knowledge of event $A$.
\end{definition}

\begin{definition}[Bayes' Theorem]\label{def:bayes-thm}
    \begin{equation*}
        P(A | B) = \frac{P(B | A) P(A)}{P(B)}
    \end{equation*}
    where $P(B) \neq 0$
\end{definition}

\subsection{Probability Distributions}

\begin{definition}[Expectation]\label{def:expectation}
    Expectation is the weighted mean of a discrete random variable $X$
    that takes on the values $x_1$, $x_2$, ... $x_n$ is:
    \begin{equation*}
        \mu_x = \sum_{i = 1}^n x_i P(X = x_i)
    \end{equation*}

    The variance of $X$ is:
    \begin{equation*}
        {\sigma_x}^2 = \sum_{i = 1}^n (x_i - \mu_x)^2 P(X = x_i)
    \end{equation*}
\end{definition}

\section{Combinatorics \& Permutations}

\begin{definition}[Fundamental Counting Principle]\label{def:fund-count-principle}
    If there are $m$ ways to do something and $n$ ways to do something else, then there are
    $m \times n$ ways to do both things.
\end{definition}

\begin{definition}[Combinations Formula]\label{def:combinations-formula}
    The total number of combinations of $r$ elements chosen from a set of $n$ possible elements is:
    \begin{equation*}
        \comb{n}{r} = \frac{n!}{r!(n-r)!}\quad
    \end{equation*}
\end{definition}

\begin{definition}[Permutations Formula]\label{def:permutations-formula}
    The total number of specific orders of possible arrangements of $r$
    elements from a set of $n$ elements is:
    \begin{equation*}
        \perm{n}{r} = \frac{n!}{(n-r)!}\quad
    \end{equation*}

    The number of permutations of $n$ objects with $n_1$ identical objects
    of one type, $n_2$ identical objects of another, and $n_k$ identical objects
    of $k$, then:
    \begin{equation*}
        \frac{n!}{n_{1!}n_{2!} ... n_{k!}}
    \end{equation*}
\end{definition}

\end{document}
