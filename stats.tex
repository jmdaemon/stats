\documentclass[11pt]{article}
\usepackage{coco}

% TODO: Remove disgusting paragraph indent
\setlength{\lineskip}{0.0pt}
\setlength{\lineskiplimit}{0pt}
\setlength{\parskip}{0pt}
\setlength{\parindent}{0pt}
\setlength{\baselineskip}{0pt}

\begin{document}

% cover page
\begin{titlepage}
    \begin{center}
        \Huge Statistics
    \end{center}
\end{titlepage}

\newpage
\tableofcontents

\section{Statistics}

Statistics is the science of conducting studies to collect, organize, summarize, analyze and draw conclusions from data.

Statistics is divided into two subfields based upon how data is used:
\begin{enumerate}
    \item Descriptive statistics: Consists of the collection, organization, summarization,
        and presentation of data. No conclusions are drawn, only data described.
    \item Inferential statistics: Consists of generalizing samples from populations,
        performing estimations, and hypothesis testing, determining relationships between
        variables, and making predictions.
\end{enumerate}

\section{Data}

\subsection{Types of Data}

% TODO: Definitions for Categorical, Numerical

% TODO: Classifications definitions for:
% Categorical: Nominal, Ordinal
% Numerical: Interval, Ratio

\subsection{Categorical}

%\begin{definition}[Categorical]\label{def:categorical}
    %Qualitative / Categorical variables have some distinct categories according to some 
%\end{definition}

\subsection{Numerical}

%\section{Classifications of Data}

%For categorical there are two classifications of data...

%For numerical there are two classifications of data...

%\section{Data Collection \& Sampling Techniques}
% TODO: Classification tables

% TODO: Parameters & Statistics
% TODO: Discrete vs Continuous variables

\section{Samples}

% Define: Population, Unit, Sample
% Quick Definitions as well as helpful table for sampling techniques
% Small subsections for each sampling technique
% Define Bias

\section{Diagrams}

% Define: Peak, Unimodal, Bimodal, Multimodal, Symmetric, Uniform, Bell-shaped, Tail, Left-Right Skews
% TODO Diagrams:
% - Ogives (Cumulative Frequency Polygons)
% - Frequency Polygyons
% - Bar Graphs, Histograms, Dot plots, Box & Whisker Plots, Grouped Frequency Distributions
% - Circle Graphs / Pie Charts (Central Angles, Percentage estimations)
% - Least Squares Regression Lines (Estimation, ensuring half data points on either side of line)
% - Stem & Leaf plots

\section{Calculations}

% TODO: Sample stdev, population stddev
% TODO: Range, Class Width, Determining class limits, class boundaries, percentiles
% TODO: Mean, Median, Mode, Midrange

% TODO: Estimated mean, estimated stddev

% TODO: Regressions, Correlation Coefficient r
% TODO: Probability

\subsection{Shorthand Calculations}

% TODO: Regressions, stddev, etc
    
\end{document}
